\documentclass[letterpaper,12pt]{article}

\textwidth 6.5in
\textheight 9in
\evensidemargin 0in
\oddsidemargin 0in
\topmargin 0in
\headheight 0in
\headsep 0in

\usepackage{graphicx}
\usepackage{color}
\usepackage{listings}
\newlength{\normalbasewidth}
\settowidth{\normalbasewidth}{\ttfamily m}
\newlength{\smallbasewidth}
\settowidth{\smallbasewidth}{\ttfamily\small m}
\newlength{\footnotebasewidth}
\settowidth{\footnotebasewidth}{\ttfamily\footnotesize m}
\newlength{\scriptbasewidth}
\settowidth{\scriptbasewidth}{\ttfamily\scriptsize m}
\lstset{ %
  basewidth=\normalbasewidth,
  basicstyle=\ttfamily\lst@ifdisplaystyle\ttfamily\normalsize\fi,
  commentstyle=\color{green},    % comment style
  escapeinside={\%*}{*)},          % if you want to add LaTeX within your code
%  frame=single,	                   % adds a frame around the code
  keepspaces=true,                 % keeps spaces in text, useful for keeping indentation of code (possibly needs columns=flexible)
  keywordstyle=\color{blue},       % keyword style
  language=C,                 % the language of the code
%  numbers=left,                    % where to put the line-numbers; possible values are (none, left, right)
%  numbersep=5pt,                   % how far the line-numbers are from the code
%  numberstyle=\tiny, % the style that is used for the line-numbers
  rulecolor=\color{black},         % if not set, the frame-color may be changed on line-breaks within not-black text (e.g. comments (green here))
  showspaces=false,                % show spaces everywhere adding particular underscores; it overrides 'showstringspaces'
  showstringspaces=false,          % underline spaces within strings only
  showtabs=false,                  % show tabs within strings adding particular underscores
%  stepnumber=1,                    % the step between two line-numbers. If it's 1, each line will be numbered
  stringstyle=\color{mymauve},     % string literal style
  tabsize=2	                   % sets default tabsize to 2 spaces
}

\newcommand{\tuple}[1]{\ensuremath{\left \langle #1 \right \rangle }}

\title{Probabilistic Robotics}
\author{Programming Assignment 1}

\begin{document}
\maketitle

The goal of this assignment is to write a range sensor model.  The
sensor model calculates $p(s|z,m)$ where $s$ is the robot state or
pose expressed as the tuple $\tuple{x,y,\theta}$, $z$ is an
observation derived from sensor data, and $m$ is an occupancy grid
map.  In this case, $z$ is a range reading from a sensor which is
mounted somewhere on the robot.  For localization to work properly, it
is necessary to account for the position of the sensor with respect to
the center of the robot, as well as the angle of the sensor beam with
respect to the front of the robot.  The following figure shows the
situation:

\begin{center}
  \input{rangesensor.pdf_t}
  \end{center}

In this figure, the large circle represents the robot and the
smaller grey circle represents a generic distance sensor unit.
$dx$ and $dy$ give the location of the sensor beam receiver
in robot-centric coordinates, and $\alpha$ gives the angle of the
beam with respect to the front of the robot.  An alpha value
of zero means that the sensor beam projects directly ahead of
the robot, while a value of ninety means that the beam is
directed to the left, orthogonal to the front of the robot.

You are to write a function to calculate $p(s|z,m)$, while
taking all of this data into account.  The function
prototype is:
\begin{lstlisting}
float sensormodel(MapStruct *map,
                  int res,
                  int row,int col,
                  int theta,
                  int dx, int dy,
                  int a,int r);
\end{lstlisting}
where \lstinline{r} is the range reading along the sensor beam.  Note
that $s$ is actually the tuple $\tuple{{\tt row},{\tt col},{\tt
    theta}}$ and $z$ is the tuple $\tuple{{\tt dx},{\tt dy},{\tt
    a},{\tt r}}$.  The map is a 2D array of unsigned 8-bit integers,
and a number, {\tt res}, indicating the size of each cell. If {\tt res}
is 100, then each cell in the map is 100 mm by 100 mm.
A cell value of 0 indicates that a cell is occupied with probability 1,
while a value of 255 indicates that the cell is occupied with
probability 0.

I have written a program to load everything and test your function.
You just have to edit {\tt sensormodel.cc} and write the function.
\lstinline{MapStruct} is defined in {\tt map.h}.  You may
need to write some helper functions.  You will definitely
need a function to simulate the sensor beam.  Its prototype will
look something like this:
\begin{lstlisting}
int simulate_beam(MapStruct *map,int res,int x,int y,double angle);
\end{lstlisting}
where {\tt x} and {\tt y}  are the position
of the sensor in map coordinates, and angle is the direction
of the beam.  It should return a result in millimeters.


\end{document}
